\documentclass[11pt,a4paper]{article}

\usepackage{titling}
\usepackage[hidelinks]{hyperref}
\usepackage{graphicx}
\usepackage{grffile}
\usepackage{float}
\usepackage{geometry}
\usepackage{listings}
\usepackage{xcolor}

\newcommand{\subtitle}[1]{
  \posttitle{
    \par\end{center}
    \begin{center}\large#1\end{center}
    \vskip0.5em}
}

\definecolor{mygreen}{rgb}{0,0.6,0}
\definecolor{mygray}{rgb}{0.5,0.5,0.5}
\definecolor{mymauve}{rgb}{0.58,0,0.82}
\definecolor{light-gray}{gray}{0.95}
\lstset{ %
  backgroundcolor=\color{light-gray},   % choose the background color; you must add \usepackage{color} or \usepackage{xcolor}
  basicstyle=\footnotesize,        % the size of the fonts that are used for the code
  breakatwhitespace=false,         % sets if automatic breaks should only happen at whitespace
  breaklines=true,                 % sets automatic line breaking
  captionpos=b,                    % sets the caption-position to bottom
  commentstyle=\color{mygreen},    % comment style
  deletekeywords={...},            % if you want to delete keywords from the given language
  escapeinside={\%*}{*)},          % if you want to add LaTeX within your code
  extendedchars=true,              % lets you use non-ASCII characters; for 8-bits encodings only, does not work with UTF-8
  frame=single,	                   % adds a frame around the code
  keepspaces=true,                 % keeps spaces in text, useful for keeping indentation of code (possibly needs columns=flexible)
  keywordstyle=\color{blue},       % keyword style
  language=Octave,                 % the language of the code
  otherkeywords={*,...},           % if you want to add more keywords to the set
  numbers=left,                    % where to put the line-numbers; possible values are (none, left, right)
  numbersep=5pt,                   % how far the line-numbers are from the code
  numberstyle=\tiny\color{mygray}, % the style that is used for the line-numbers
  rulecolor=\color{black},         % if not set, the frame-color may be changed on line-breaks within not-black text (e.g. comments (green here))
  showspaces=false,                % show spaces everywhere adding particular underscores; it overrides 'showstringspaces'
  showstringspaces=false,          % underline spaces within strings only
  showtabs=false,                  % show tabs within strings adding particular underscores
  stringstyle=\color{mymauve},     % string literal style
  tabsize=2,	                   % sets default tabsize to 2 spaces
  title=\lstname                   % show the filename of files included with \lstinputlisting; also try caption instead of title
}

\begin{document}
\title{Benchmark Service Installation Manual}
\subtitle{ Git: \url{https://github.com/CodingInfinity/Benchmark-Service-Documentation} \\ GitHub Organisation: \url{https://github.com/CodingInfinity}}
\begin{figure}
      \centering
      \includegraphics[height=230px]{../Images/CodingInfinity.png}
\end{figure}
\author{
	\textbf{The Client:} \\
	Ms Vreda Pieterse  \\
	Department of Computer Science \\
	University Of Pretoria
	\\
	\\
	\textbf{The Team:} \\
	Andrew Broekman		\emph{11089777}	\\
	Brenton Watt		\emph{14032644}	\\
	Fabio Loreggian		\emph{14040426}	\\
	Reinhardt Cromhout	\emph{14009936}	\\
}
\date{\textbf{September 2016}}

\maketitle
\thispagestyle{empty}
\pagebreak

\tableofcontents
\pagebreak

\section{Introduction}
This is the installation manual for the Benchmark Service. It details how to
install each part of the system in order to provide this service. An installed
prototype of this service is hosted at the Department of Computer Science at
the University of Pretoria.

\section{Benchmark Management Server}
The first thing is to make sure that you have Maven set up and working on
the target machine.\\

\noindent Then the repository can be cloned using git with this command:
\begin{lstlisting}[language=bash]
$ git clone https://github.com/CodingInfinity/Benchmark-Management-Server.git
\end{lstlisting}
Thereafter the desired properties can be set in the application-prod.yml file
which can be found in the src/main/resources/config directory.\\ \\
\noindent The server can then be run by executing the terminal command:
\begin{lstlisting}[language=bash]
$ spring-boot:run
\end{lstlisting}

\subsection{Required Dependencies}
The Benchmarking system requires a PostgreSQL and Apache ActiveMQ setup.
A docker-compose.yaml file, located in the root of the Management Server, can
be used to provide the required infrastructure.

\section{Benchmark Web Interface}
On the machine which will be serving as the web host, the following commands can
be used to compile the Angular2 front-end application:
\begin{lstlisting}[language=bash]
$ git clone https://github.com/CodingInfinity/Benchmark-Web-Application.git
$ cd Benchmark-Web-Application
$ npm install
$ npm build
\end{lstlisting}

After the above commands have been issued, a \textit{dist} folder can be located
within the project root directory. This folder can be servered as the root
directory with a web server such as Apache HTTPd or nginx.

\section{Benchmark Instrumentation Application}
This section details the installation steps required to setup the
instrumentation application on the profiler node.
\subsection{Required Dependencies}
\begin{itemize}
  \item Relative POSIX-compliant *NIX system
  \item g++ 4.2
  \item boost 1.53.0
  \item Runtime libraries for lex and yacc might be needed for the Apache Thrift compiler.
\end{itemize}

\subsection{External Dependencies}
\begin{itemize}
  \item Apache Qpid Proton
  \item Apache Qpid CPP with AMQP version 1.0 (Requires Apache Qpid Proton)
  \item Apache Thrift C++ binding
  \item Hyperic SIGAR
  \item YAML CPP
\end{itemize}

For all dependencies listed below please refer to relevant documentation for
building the dependency in question, such as required dependencies required by
the libraries itself.

\subsubsection{Apache Qpid Proton}
\begin{lstlisting}[language=bash]
$ git clone git://git.apache.org/qpid-proton.git
$ cd qpid
$ mkdir cmake-build
$ cd cmake-build
$ cmake ..
$ make all
$ make install
\end{lstlisting}

\subsubsection{Apache Qpid CPP}
\begin{lstlisting}[language=bash]
$ git clone git://git.apache.org/qpid-cpp.git
$ cd qpid-cpp
$ mkdir cmake-build
$ cd cmake-build
$ cmake ..
$ make all
$ make install
\end{lstlisting}

\subsubsection{Apache Thrift}
\begin{lstlisting}[language=bash]
$ git clone git://git.apache.org/thrift.git
$ cd thrift
$ mkdir cmake-build
$ cd cmake-build
$ cmake ..
$ make
$ make install
\end{lstlisting}

\subsubsection{Hyperic SIGAR}
\begin{lstlisting}[language=bash]
$ git clone https://github.com/hyperic/sigar.git
$ cd siger
$ mkdir cmake-build
$ cd cmake-build
$ cmake ..
$ make
$ make install
\end{lstlisting}

\subsubsection{YAML CPP}
\begin{lstlisting}[language=bash]
$ git clone https://github.com/jbeder/yaml-cpp
$ cd yaml-cpp
$ mkdir cmake-build
$ cd cmake-build
$ cmake ..
$ make
$ make install
\end{lstlisting}

To compile the Instrumentation software on the profiler node, the following
commands can be issued from a terminal:
\begin{lstlisting}[language=bash]
$ git clone https://github.com/CodingInfinity/Benchmark-Instrumentation-Application.git
$ cd Benchmark-Instrumentation-Application
$ mkdir cmake-build
$ cd cmake-build
$ cmake ..
$ make
$ make install
\end{lstlisting}

Upon completion, the compiled application, \textit{Instrumentation}, can be
found within the cmake-build directory. 

\end{document}
