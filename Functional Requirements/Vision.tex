There are many cases in the industry, in research and in education where benchmarking of
a software product is important. In industry benchmarks can be used to ensure that soft-
ware performance is acceptable. In the competitive commercial world, benchmarks can
serve to inform buyers of comparative performance of competing products. The bench-
marking forms an integral part of research related to the development of new algorithms
and techniques as well as the refinement and optimization of existing operations. Algo-
rithm and data structure benchmarks can be applied as a very useful teaching tool for
students to review the notions of space and time complexity.\\ \\
It is strange that although benchmarking seem to be a very common and useful applica-
tion, very few benchmarking tools or services are available. Those that are available re-
quires intricate configuration that may be beyond the reach of the developers, researchers,
teachers and students who would like to use them. The development of a benchmark-
ing service which can be used in a generic way would therefore be welcomed by a large
potential user base.
