\subsection{Background in Benchmarking}
There are many cases in the industry, in research and in education where benchmarking of
a software product is important. In industry benchmarks can be used to ensure that software
performance is acceptable. In the competitive commercial world, benchmarks can serve to inform
buyers of comparative performance of competing products. The benchmarking forms an integral
part of research related to the development of new algorithms and techniques as well as the refinement
and optimization of existing operations. Algorithm and data structure benchmarks can be applied
as a very useful teaching tool for students to review the notions of space and time complexity.

\subsection{Our System}
Our system will consist of \textbf{five} modules namely:
\begin{itemize}
	\item Notification 
	\item Reporting
	\item Repository Management
	\item Test Management
	\item User Management.
\end{itemize}
The following modules may contain, and are not limited to the following, as according to Agile Process.


\subsubsection{Notification}
This module consists of:
\begin{itemize}
	\item Notifying the user that their Benchmark is complete
\end{itemize}

\subsubsection{Reporting}
This module consists of:
\begin{itemize}
	\item Reporting of a Benchmarking instance
	\item Comparing multiple Benchmarks
	\item Exporting to report PDF
\end{itemize}

\subsubsection{Repository Management}
This module consists of:
\begin{itemize}
	\item Algorithm Category Management
	\item Dataset Category Management
	\item Algorithm Management
	\item Dataset Management
\end{itemize}

\subsubsection{Test Management}
This module consists of:
\begin{itemize}
	\item Providing basic template data
	\item Adding data to repository
	\item Generating Test data
	\item Storing Test Data
\end{itemize}

\subsubsection{User Management}
This module consists of:
\begin{itemize}
  \item User Admin
  \item User Signup
  \item User Login
  \item User Profile Management
\end{itemize}



