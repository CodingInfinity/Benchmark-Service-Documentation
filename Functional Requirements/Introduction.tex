There are many cases in the industry, in research and in education where benchmarking of
a software product is important. In industry benchmarks can be used to ensure that software
performance is acceptable. In the competitive commercial world, benchmarks can serve to inform
buyers of comparative performance of competing products. The benchmarking forms an integral
part of research related to the development of new algorithms and techniques as well as the refinement
and optimization of existing operations. Algorithm and data structure benchmarks can be applied
as a very useful teaching tool for students to review the notions of space and time complexity.
