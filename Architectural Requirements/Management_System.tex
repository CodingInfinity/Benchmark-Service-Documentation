This section specifies the software architecture requirements and the software
architecture design at the second level of granularity - the management service
of the benchmarking system. The management service will be responsible for all
administration of user management, repository management, experiment management
and reporting. The management service will communicate with the client monitor 
applications through the messaging platform. The reader is referred to
figure \ref{fig:managementSoftwareArchitecture} for an overview of the
management platform architecture and requirements. 

\section{Overall Software Architecture}
\subsection{Architecture Requirements}
This section discusses the software architecture requirements around the
back end management system infrastructure. The back end management system will be
responsible for the delegation of jobs to cluster nodes, persisting of results,
user management and allowing users to derive value from reporting. In particular
the architecture requirements at the second level of granularity is specified.

\subsubsection{Access and Integration Requirements}
\label{sec:accessIntegrationRequirementsManagementSystem}
\paragraph*{Access Channels}
The access channels can be divided into two broad categories based on the user type
\begin{itemize}
	\item Human Access Channel
	\item System Access Channel
\end{itemize}

\subparagraph*{Human Access Channel}
\label{sec:humanAccessChannelManagementSystem}
The system will be providing a REST based access channel to be used by human
users via the HTML 5/JavaScript Single Page Application Web Interface.

\subparagraph*{System Access Channel}
The system will expose a further access channel to be used by the so-called
``Benchmark Client" or ``monitor" application. This access channel will be
utilizing a message bus architecture to deliver messages between the client
and the management back end system.

This access channel will however not be accessible to end users.

\paragraph*{Integration Channels}
The various integration channels of the benchmarking system
\begin{itemize}
	\item Integration with a persistence provider.
	\item Integration with the human access channels, namely the web interface.
	\item Integration with the message platform architecture.
\end{itemize}

The integration with the persistence provider is required as we need to persist
the measurements obtained with the benchmarking client. Further more as per the
client's request, test data must also be persisted as it is envision that a
repository of test data will be build.

In order to make the results from the benchmarking tests useful, users will
need some way to interact and manipulate the data in order to be able to
derive value from this data. To enable users to interact with the data,
a web interface will be provided.

The final integration required by the management system is that of integration
with the message platform. In order to better decouple the monitor and
management systems, a messaging architecture was introduced. The management
system will process the results from a queue structure managed by the messaging
system. 

Once a program is uploaded via the a human access channel, it will be deployed
to an autonomous monitor node upon which the benchmarking will commence.  The 
reason for utilizing a messaging architecture is assist in making nodes autonomous
in order to meet the security quality requirements as provided in
section \ref{sec:securityQualityRequirement}.

\subsubsection{Quality Requirements}
\label{sec:qualityRequirementManagementSystem}
The quality requirement are the requirements around the quality attributes of
the systems and the services it provides. This includes requirements like
maintainability, flexibility, extensibility, performance, scalability, security,
auditability, usability and testability requirements.


\paragraph*{Authentication}
The system needs to support a simple registration and authentication framework
which will determine what each user can do based on their authority level.

\paragraph*{Flexibility}
Persistence architectures and reporting infrastructures are rapidly evolving as can
be seen from the rapid growth of NoSQL databases, semantic knowledge repositories and big data
stores. In this context it is important that the application functionality is not locked into any
specific persistence technology and that one is able to easily modify the persistence provider and
reporting framework.

\paragraph*{Testability}
All services offered by the system must be testable through
\begin{enumerate}
	\item automated unit tests testing components in isolation using mock objects, and
	\item automated integration tests where components are integrated within the actual environment.
\end{enumerate}

In either case, these functional tests should verify that
\begin{itemize}
	\item the service is provided if all pre-conditions are met (i.e. that no exception is raised)
	\item the correct exception is thrown when the corresponding pre-condition
	is violated.
	\item that all post-conditions hold true once the service has been provided.
\end{itemize}

In addition to functional testing, the quality requirements should also be tested.

\subsection{Architecture Design}
This section specifies the software architecture design for the second level of 
granularity. It includes the allocation of architectural responsibilities to 
architectural components, any tactics which should be used at the current level
of granularity to address quality requirements,

\subsubsection{Architectural Responsibilities and Components}
Figure \ref{fig:architectureResponsibilityAllocation} shows the allocation of
architectural responsibilities to architectural components. 
\begin{figure}[H]
	\begin{center}
	\includegraphics[scale=0.5]{../Diagrams and Charts/Architecture/ResponsibilityAllocation.jpg}
	\caption{The abstract architectural components to which the architectural responsibilities are
assigned.}
  \label{fig:architectureResponsibilityAllocation}
	\end{center}
\end{figure}

\section{Database}
The database is a persistence provider providing long term, organized storage
of the data which will be generated through the application. Various types of
database exist, which are mainly categorized based on the approach with which
the provider in questions represents and stores the data internally. Some of 
the types of databases which where considered for this project include
\begin{itemize}
	\item Relational Databases e.g. MySQL, PostgreSQL
	\item Document-oriented Databases e.g. MongoDB, Couchbase
	\item Graph-based Databases Neo4j, OrientDB
\end{itemize}

In choosing a persistence provider, it is important to consider the data as
well as the relationship between the data to ensure the correct type of 
provider is selected.

The data to be generated by the benchmarking service will have a very flat
relationship structure between data elememtns. This flat structure naturally
makes the data well suited for a document-based persistence provider.

\subsection{Architecture Requirements}
\subsubsection{Access and Integration Requirements}
\subsubsection{Quality Requirements}
\subsubsection{Architectural Responsibilities}
\subsubsection{Architecture Constraints}
\subsection{Architecture Design}
\subsubsection{Tactics}
\subsubsection{Architectural Components}
\subsubsection{Frameworks and Technologies}
\paragraph{Concrete Realization of Architectural Components}
\paragraph{Tactics}
\paragraph{Tools}
\paragraph{Concepts and Constraints for Application Components}

\section{Persistence API}
The persistence API provides abstracted access to a persistence provider while
remaing decoupled from the underlying technology, including but not limited to
database, SQL version and transaction management, whilst employing a range of
software engineering tactics to concretely address required quality
requirements required for the peristence domain.

\subsection{Architecture Requirements}
The architectural requirements for the persistence API include the refined
quality requirements and architectural requirements listed below. The
architectural constraints for this lower level components are the same as for
the system as whole, as referred to in 
section \ref{sec:systemArchitecturalConstraints} with further extensions as
specified in section \ref{sec:persistenceAPIArchitecturalConstraints}.

\subsubsection{Access and Integration Requirements}
\subsubsection{Quality Requirements}
\paragraph{Flexibility}
The provided persistence API should be able to adapt to the rapidly evolving
persistence architecture domain, especially in terms of the different 
methodologies of storing data such as relational and NoSQL data stores. It is
further import that the persistence layer is not locked to any specific
persistence technology.

Further the chosen API should to force the use of any vender specific
transaction management, but should rather provide an abstraction layer allowing
the use of any transaction manager implementing the required interface.

\paragraph{Maintainability}
\label{sec:persistenceAPIMaintainability}
The used persistence API should be in a mature stage of the software development
lifecycle as to guard against a rapidly evolving changing API. The chosen
persitence API should be an open standard with multiple realization as to guard
against realization techologies be aboned. This will allow in future an easier 
switch to another persistence API implementaion if required for the long term
maintance and use of the project.

\paragraph{Scalability}
The chosen persistence API should be able to allow for future scaling of the
infrastrure either horizontally or vertically with a preference for 
horizontal scaling.

\paragraph{Performance}
The persitence API should allow for the use of certain architectural tactics
to increase performance. Specifically the following tactics should be supported
to some extent
\begin{itemize}
	\item Object Caching
	\item Connection Pooling
	\item Thread Provisioning
	\item Scheduling
\end{itemize}

\paragraph{Reliability}
The chosen API should allow security at least providing authorization on
entities managed by the underlying persistence technology architecture.

\subsubsection{Architectural Responsibilities}
The architectural responsibilities of the persistence API are shown in 
Figure \ref{fig:persistenceResponsibilities}
\begin{figure}[H]
	\begin{center}
	\includegraphics[scale=0.5]{../Diagrams and Charts/Persistence API/Responsibilities.jpg}
	\caption{The architectural responsibilities of the Persistence API}
	\label{fig:persistenceResponsibilities}
	\end{center}
\end{figure}

\subsubsection{Architecture Constraints}
The chosen persistence API should be a currently active standard, with a medium
to large sized active community supporting the standard and should be realized
by at least three active realizations of the chosen API as to ensure future 
maintainability as setout by the required quality requirement in 
section \ref{sec:persistenceAPIMaintainability}.
\label{sec:persistenceAPIArchitecturalConstraints}

\subsection{Architecture Design}
\subsubsection{Tactics}
The persistence API implement the following tactics:
\begin{itemize}
	\item \textit{Object Relational Mapping} to reduce code bulk, improve
		maintainability and allow for decoupling from the persistence 
		provider.
	\item \textit{Query Mapping} from queries across a graph of Java objects
		onto the database queries used in the selected database 
		technology and provider.
	\item \textit{Object caching} to imporve scalability and performance.
	\item \textit{Transactions} with 2-phase commit to improve reliability
		of processes.
	\item \textit{Transaction Neutral API} allow the use of any transaction manager
		implementing the required interface as to improve flexibility and
		future maintainability.
	\item \textit{Connection Pooling} to improve performance and scalability.
\end{itemize}

\subsubsection{Architectural Components}
The architectural components of the persistence API are shown in Figure \ref{fig:persistenceResponsibilityAllocation}
\begin{figure}[H]
	\begin{center}
	\includegraphics[scale=0.5]{../Diagrams and Charts/Persistence API/ResponsibilityAllocation.jpg}
	\caption{The abstract components to which the architectural responsibilities are assigned.}
	\label{fig:persistenceResponsibilityAllocation}
	\end{center}
\end{figure}

\subsubsection{Frameworks and Technologies}
A JPA 2.1 (Java Persistence API Version 2.1) provider will be used as a 
persistence API. The chosen concrete implementation used will be Hibernate as
it has support for both relational and NoSQL persistence providers.

The JPA 2.1 API is a widely supported public standard which are implemented by
the following products:
\begin{itemize}
	\item Hibernate
	\item EclipseLink
	\item DataNucleus
\end{itemize}

Further more the listed implementations all support the use of NoSQL databases
by utilizing the JPA standard for relational databases. Thus the required
quality requirements of using an open standard with multiple implmentations
supporting both relational and NoSQL database are fullfilled.

The persistence context (EntityManager) will will be dependency injected into 
services requiring access to persistent data. JPA providers do implement
\begin{itemize}
	\item \textit{Object Relational Mapping} including the mapping of
		relationships between objects via a provided ORM 
		implementation such as Hibernate or EclipseLink.
	\item \textit{Query Mapping} from object-oriented queries across the
		domain object graph to queries for a specific database provider.
	\item \textit{Object caching} within the persistence context with
		in-memory or NoSQL database based caching allow for the 
		fulfillment of the performance, flexibility and scalability
		quality requirements.
	\item \textit{Transactions} are supported through the use of the \textit{Java Transaction API (JTA)}.
	\item \textit{Transaction Neutral API} is supported throught the Spring 
		Framework transaction manager supporting a consistent programming
		model across different transaction APIs such as 
		Java Transaction API, JDBC, Hibernate, Java Persistence API and 
		Java Data Objects.
	\item \textit{Connection Pooling} is provided through a JCA connector 
		based implementation of a JDBC driver.
\end{itemize}

Queries will be specified as Spring Data JPA queries, thereby reducing 
boilerplate code which ease future maintenace and development, as the queries
are not specific to any underlying query language e.g. SQL or any underlying 
persistence technology such as relational or NoSQL providers.

\paragraph{Concrete Realization of Architectural Components}
\paragraph{Tactics}
\paragraph{Tools}
\paragraph{Concepts and Constraints for Application Components}
The application concepts wihtin the persistence domain include
\begin{itemize}
	\item \textit{Domain objects} which host long-living state objects, and is realized in the Java architecure as Plain Old Java Objects (POJO's) which doesn't contain any business logic.
	\item \textit{Queries across object graph of domain objects} through which the required information of state in the domain objects is retrieved, modified and removed.
\end{itemize} 

\section{Web Services Framework}
The web services framework is used to expose business services in a
technology-neutral way over some network, which in most cases is the public
Internet.  Wrapping business services in a technology-neutral layer allows
one to decouple the front-end technologies, specifically the user interface
technologies from the back-end technologies, it also allows for the decoupling
of back-end services from one another, in effect communication between dispate
applications.  This decoupling provides one with the ability to vary either the
front-end technologies and back-end technologies independantly from one
another.  Furthmore this allows one to write back-end servies in the most
appropriate technology stack and then have seamless communication between these
indiivdual components.

\subsection{Architecture Requirements}
\subsubsection{Access and Integration Requirements}
\subsubsection{Quality Requirements}
\subsubsection{Architectural Responsibilities}
\subsubsection{Architecture Constraints}
\subsection{Architecture Design}
\subsubsection{Tactics}
\subsubsection{Architectural Components}
\subsubsection{Frameworks and Technologies}
\paragraph{Concrete Realization of Architectural Components}
\paragraph{Tactics}
\paragraph{Tools}
\paragraph{Concepts and Constraints for Application Components}

\section{Web Application Framework}
\subsection{Architecture Requirements}
\subsubsection{Access and Integration Requirements}
\subsubsection{Quality Requirements}
\subsubsection{Architectural Responsibilities}
\subsubsection{Architecture Constraints}
\subsection{Architecture Design}
\subsubsection{Tactics}
\subsubsection{Architectural Components}
\subsubsection{Frameworks and Technologies}
\paragraph{Concrete Realization of Architectural Components}
\paragraph{Tactics}
\paragraph{Tools}
\paragraph{Concepts and Constraints for Application Components}

\section{Reporting}
The Reporting module represents the core of the value that the users will
get out of the system. Reports are the means with which the users will
be able to see the results of the benchmarking of their programs.

\subsection{Scope}
The scope for the reporting module is shown in Figure \ref{fig:reportingScope}
\begin{figure}[H]
  \begin{center}
  \includegraphics[scale=0.38]{../Diagrams and Charts/Reporting/Scope.jpg}
  \caption{Reporting Scope}
  \label{fig:reportingScope}
  \end{center}
\end{figure}


\subsection {Download Results}
The \textit{Download Results} use case is concerned with presenting the results
of a job to the user in a global interchange format, which was chosen
as CSV (Comma Seperated Value) representation.

\subsubsection{Service Contract}
The service contract for downloading results for a specified job is shown in 
Figure \ref{fig:downloadResultsServiceContract}
\begin{figure}[H]
  \begin{center}
  \includegraphics[scale=0.38]{../Diagrams and Charts/Reporting/Download Results Service Contract.jpg}
  \caption{Download Results Service Contract}
  \label{fig:downloadResultsServiceContract}
  \end{center}
\end{figure}



