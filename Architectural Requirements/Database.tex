The database is a persistence provider providing long term, organized storage
of the data which will be generated through the application. Various types of
database exist, which are mainly categorized based on the approach with which
the provider in questions represents and stores the data internally. Some of 
the types of databases which where considered for this project include
\begin{itemize}
	\item Relational Databases e.g. MySQL, PostgreSQL
	\item Document-oriented Databases e.g. MongoDB, Couchbase
	\item Graph-based Databases Neo4j, OrientDB
\end{itemize}

In choosing a persistence provider, it is important to consider the data as
well as the relationship between the data to ensure the correct type of 
provider is selected.

The data to be generated by the benchmarking service will have a very flat
relationship structure between data elememtns. This flat structure naturally
makes the data well suited for a document-based persistence provider.

\subsection{Architecture Requirements}
\subsubsection{Access and Integration Requirements}
\subsubsection{Quality Requirements}
\subsubsection{Architectural Responsibilities}
\subsubsection{Architecture Constraints}
\subsection{Architecture Design}
\subsubsection{Tactics}
\subsubsection{Architectural Components}
\subsubsection{Frameworks and Technologies}
\paragraph{Concrete Realization of Architectural Components}
\paragraph{Tactics}
\paragraph{Tools}
\paragraph{Concepts and Constraints for Application Components}