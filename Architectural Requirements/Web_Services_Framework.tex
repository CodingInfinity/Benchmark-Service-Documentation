The web services framework is used to expose business services in a
technology-neutral way over some network, which in most cases is the public
Internet.  Wrapping business services in a technology-neutral layer allows
one to decouple the front-end technologies, specifically the user interface
technologies from the back-end technologies, while simultaneously allowing for the decoupling
of back-end services from one another, in effect communication between disparate
applications.  This decoupling provides one with the ability to vary either the
front-end technologies and back-end technologies independently from one
another.  Furthermore this allows one to write back-end services in the most
appropriate technology stack and then have seamless communication between these
individual components.

\subsection{Architecture Requirements}
\subsubsection{Access and Integration Requirements}
The web service framework serves as the bridge between the client and back end
systems, allowing one to communicate with the other. For this reason it is
important the access channel to be used between the client and back end systems
should utilize common and standard compliant protocols to ensure the greatest
amount of integration can be achieved to provide users with maximum value.

\subsubsection{Quality Requirements}
\paragraph*{Maintainability}
\label{sec:webServicesFrameworkMaintainability}
The web services framework is concerned with wrapping business logic, thereby
allowing one to categorize this as so called "plumbing code" which should be as
far as possible be removed from the actual code. This code is normally applied
by the use of annotations in the Java context or by weaving the code into
existing business code using aspects.

Using the above mentioned approach allows one more easily to maintain the code
base.

\paragraph*{Integrability}
The framework should enable one to expose business services in a technology
neutral way to allow for easy integration with other independent business
services and systems.  The chosen technology neutral format should be supported
by the business services and system that require integration.



\subsection{Architecture Design}
\subsubsection{Architectural Responsibilities, Components and Realization}
The concrete components addressing the required responsibilities are shown in Figure \ref{fig:webServicesFrameworkResponsibilityRealization}.
\begin{figure}[H]
	\begin{center}
	\includegraphics[scale=0.5]{../Diagrams and Charts/Web Services Framework/ResponsibilityRealization.jpg}
	\caption{The components within Jersey addressing the architectural responsibilities of the Web Services Framework}
	\label{fig:webServicesFrameworkResponsibilityRealization}
	\end{center}
\end{figure}

\subsubsection{Tactics}
The Web Services Framework implement the following tactics:
\begin{itemize}
\item \textit{Support Communication Channels}
The web services framework should support the communication channel to be used
between the client and server as well as between between business services.
With regards to this project the communication channels will consists of a
standards based network connection between all parties, with the communication
channel not necessarily being uniform between all parties. The most likely
communication channel between the client and server will be the Internet
network with a internet network between business services.

\item \textit{Support Standard Communication Protocols}
The web service framework should support standards based communication
protocols as this will ensure the highest change of ensuring full
integrability between client and other business services. All parties should
at least support one of the standards the selected web services framework
supports, thereby ensuring all parties will have successful communication.
\end{itemize}


\subsubsection{Frameworks and Technologies}
We decide upon a light, text-based communication standard, namely REST, which
will be used to bridge the communication between the client and back end systems.
Various frameworks exists, however the Java programming language exposes a
vendor neutral REST wrapping API referred to as JAX-RS. The advantage of using a
vendor neutral API is that there are various realizations of these API.

The API assist in reducing boilerplate code by weaving in code with the use of
annotations, which assists in achieving required quality requirements namely
maintainability and flexibility.

For the project in question, we will be using the Jersey reference implementation
from Sun of the JAX-RS API standard.
