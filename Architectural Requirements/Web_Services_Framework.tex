The web services framework is used to expose business services in a
technology-neutral way over some network, which in most cases is the public
Internet.  Wrapping business services in a technology-neutral layer allows
one to decouple the front-end technologies, specifically the user interface
technologies from the back-end technologies, it also allows for the decoupling
of back-end services from one another, in effect communication between dispate
applications.  This decoupling provides one with the ability to vary either the
front-end technologies and back-end technologies independantly from one
another.  Furthmore this allows one to write back-end servies in the most
appropriate technology stack and then have seamless communication between these
indiivdual components.

\subsection{Architecture Requirements}
The architectural requirements for the web service framework include the
refined quality requirements and architectural requirements listed below. The
architectural constraints for this lower level components are the same as for
the system as whole, as referred to in section \ref{sec:systemArchitecturalConstraints}
with further extensions as specified in section \ref{sec:persistenceAPIArchitecturalConstraints}.

\subsubsection{Access and Integration Requirements}
\subsubsection{Quality Requirements}
\subsubsection{Architectural Responsibilities}
\subsubsection{Architecture Constraints}
\subsection{Architecture Design}
\subsubsection{Tactics}
\subsubsection{Architectural Components}
\subsubsection{Frameworks and Technologies}
\paragraph{Concrete Realization of Architectural Components}
\paragraph{Tactics}
\paragraph{Tools}
\paragraph{Concepts and Constraints for Application Components}