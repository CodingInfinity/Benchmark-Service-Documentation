This section will specify the quality requirements at the highest level of
granularity. These requirements will propagate throughout the entire system
into every other compopent. Quality requirements that are spesific to certain
parts of the system will be discussed under that system's section.
\paragraph{Security}
\subparagraph{Authentication}
The system needs to support a simple registration and authentication framework
which will determine what each user can do based on their authority level. This
also involves users at a higher level of authority having the ability to manage
the users at a level of authority lower than theirs.
\subparagraph{Flexibility}
Persistence architectures and reporting infrastructures are rapidly evolving as can
be seen from the rapid growth of NoSQL databases, semantic knowledge repositories and big data
stores. In this context it is important that the application functionality is not locked into any
specific persistence technology and that one is able to easily modify the persistence provider and
reporting framework.\\
Futhermore it is important that the Client layer can easily be swapped out for a different
layer that accesses the system in a different why using the layer beneath it.
\begin{itemize}
 \item The system needs to be able to handle high amounts of concurrent traffic.
 \item The system needs to successfully queue requests in such a way that no
	   benchmarking request gets lost in high traffic volume.
 \item The product will need a persistent database that will hold user results
       for a set period of time as well as user data and generated test data to
       be used in the future. 
\end{itemize}
