This section will specify the quality requirements at the highest level of
granularity. These requirements will propagate throughout the entire system
into every other compopent. Quality requirements that are spesific to certain
parts of the system will be discussed under that system's section.
\paragraph{Security}
\subparagraph{Authentication}
The system needs to support a simple registration and authentication framework
which will determine what each user can do based on their authority level. This
also involves users at a higher level of authority having the ability to manage
the users at a level of authority lower than theirs.
\paragraph{Flexibility}
Persistence architectures and reporting infrastructures are rapidly evolving as can
be seen from the rapid growth of NoSQL databases, semantic knowledge repositories and big data
stores. In this context it is important that the application functionality is not locked into any
specific persistence technology and that one is able to easily modify the persistence provider and
reporting framework.\\
Futhermore it is important that the Client layer can easily be swapped out for a different
layer that accesses the system in a different why using the layer beneath it.
\paragraph{Maintainability}
Amongst the most important quality requirements for the system is
maintainability. It should be easy to maintain the system in the future. To this end

\begin{itemize}
\item future developers should be able to easily understand the system,
\item the technologies chosen for the system an be reasonably expected to be available for a long
time,
\item and developers should be able to easily and relatively quickly
	\begin{itemize}
		\item change aspects of the functionality the system provides, and
		\item add new functionality to the system.
	\end{itemize}
\item it should be easy to modify reports and add reports new reports to the system.
\end{itemize}

\paragraph{Scalability}
The system needs to be able to handle high amounts of concurrent traffic.
In addition to this the system should also be protected against
denial of service attacks why the way the system handels concurrent traffic.

\paragraph{Testability}
All services offered by the system must be testable through
\begin{enumerate}
	\item automated unit tests testing components in isolation using mock objects, and
	\item automated integration tests where components are integrated within the actual environment.
\end{enumerate}

In either case, these functional tests should verify that
\begin{itemize}
	\item the service is provided if all pre-conditions are met (i.e. that no exception is raised)
	\item the correct execption is throw when the corresponding pre-condition
	is violated.
	\item that all post-conditions hold true once the service has been provided.
\end{itemize}

In addition to functional testing, the quality requirements should also be tested.

\paragraph{Usability}
The system should be intuitive and efficient to use. Computer literacy can be
assumed, but the goal of the system is to be an easy to use Benchmarking
Service that does not require complex configuration. As such users should
not experience much difficulty in using the interface. Error messages should
also be self-explanatory.

\paragraph{Integrability}
The system should be able to easily address future integration requirements
by providing access to its services using widely adopted public standards.
All use cases which are available to human users should also be accessible from external systems.

\paragraph{Deployability}
\label{sec:systemDeployability}


\paragraph{Initial discussion added before demo 1}
\begin{itemize}
 \item The system needs to be able to handle high amounts of concurrent traffic.
 \item The system needs to successfully queue requests in such a way that no
	   benchmarking request gets lost in high traffic volume.
 \item The product will need a persistent database that will hold user results
       for a set period of time as well as user data and generated test data to
       be used in the future. 
\end{itemize}
