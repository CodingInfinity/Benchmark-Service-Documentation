This section specifies the software architecture requirements and the software
architecture design for web application framework.

\subsection{Architecture Requirements}
The web application framework provides the software architecture for the software
providing browser based access to human users.

\subsubsection{Access and Integration Requirements}
\paragraph{Access Channel Requirements}
The web application framework will address the first human access channel requirment
referred to in section \ref{sec:humanAccessChannelManagementSystem},
namely the requirement of human users of the system to access the application
via a web browser. The web application must be accessable to any user who uses
the latest version of any of the major bowsers namely Mozilla Firefox, 
Google Chrome, Apple Safari, Opera and Microsoft Internet Explorer.

\paragraph{Integration Channel Requirements}
The web application must integrate with the management system in order to allow
users to derive value. However, the web application will integrate with the
business layer through the web services layer, allowing one to decouple the
business and presentation layers from one another. This assist in delivering on
the required performance, maintainability and flexability requirement set for
the system as a whole.

\subsubsection{Quality Requirements}
The most important quality requirements for the web application layer are
\paragraph{Usability}
As the system is build for human users, it is important that users using the
system feel as though the system delivers what they require when they
require it i.e. users should be allowed to derive the exact value they require
from the system, no more and no less.

\paragraph{Maintainability}
It is important that the system as a whole can be easily expanded, implying
that if the backend system is expanded the front-end should continue to function,
which will allow developers to update the frontend as required. For this reason
a modern JavaScript framework such as ReactJS was chosen to assist developers 
to maintain the system much easier, allowing the code base to mature with time
without losing strict control and structure.

\paragraph{Performance}
ReactJS improves upon the work of earlier frameworks such as AngularJS which both
users and developers felt to be slow, because of the way in which monitoring of
bounded data was implemented. With ReactJS being built from the ground up with a
focus on user expierence from Facebook Inc., a much friendlier and faster framework
was developed, with significant performance increase seen from earlier frameworks.

	
\subsubsection{Architectural Responsibilities}
The architectural responsibilities of the Web Application Framework are shown in 
Figure \ref{fig:webApplicationFrameworkResponsibilities}
\begin{figure}[H]
	\begin{center}
	\includegraphics[scale=0.35]{../Diagrams and Charts/Web Application Framework/Responsibilities.jpg}
	\caption{The architectural responsibilities of the Web Application Framework}
	\label{fig:webApplicationFrameworkResponsibilities}
	\end{center}
\end{figure}

\subsection{Architecture Design}
This section specifies the very software architecture design, i.e.
the software architecture design for a third level of granularity component, namely
the web interface. The architecture design include the allocation of architectural
responsibilities to architectural components, as well as tactics used to realize these
responsibilities under the  current level of granularity to address stated
quality requirements.

\subsubsection{Tactics}
Tactics ReactJS uses in order to address the quality requirements should include:
\begin{itemize}
	\item caching of pre-generated and pre-populated HTML pages for performance
	\item virtual DOM for in-memory updates, incremental builds and efficient 
	diffing based on differentiation between static and dynamic DOM elements
\end{itemize}

\subsubsection{Architectural Components}
The architectural components of the  Web Application Framework are shown in Figure \ref{fig:webApplicationFrameworkResponsibilityAllocation}
\begin{figure}[H]
	\begin{center}
	\includegraphics[scale=0.35]{../Diagrams and Charts/Web Application Framework/ResponsibilityAllocation.jpg}
	\caption{The abstract components to which the Web Application Framework responsibilities are assigned.}
	\label{fig:webApplicationFrameworkResponsibilityAllocation}
	\end{center}
\end{figure}

\subsubsection{Frameworks and Technologies}
The web framework the project will be utilizing will be based on the ReactJS
which is a JavaScript framework developed by Facebook Inc. ReactJS is an open-source web
application framework which promises a one-way flow of information between 
model, controller and view. This promise is meant to increase performance,
decoupling and maintainablity of the underlying code base.

Core reasons for using ReactJS include:
\begin{itemize}
	\item Framework requires a defined structure which aids in maintainability.
	\item The framework is maintained not only by a community of users but also
	by Facebook and Instagram, which aids in maintainability due to the 
	constant upkeep of the code base and documentation.
	\item ReactJS supports very good integration with RESTful API services.
	\item Release of React Native which will allow for React based architecture 
	to native Android and iOS applications making the task of developing an 
	application easier as only one code base needs to be developed and maintained.
\end{itemize}

Other frameworks that were considered;
\begin{itemize}
	\item AngularJS
	\item Ember.js 
	\item Backbone.js
\end{itemize}