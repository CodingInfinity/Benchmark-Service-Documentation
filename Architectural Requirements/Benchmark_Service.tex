This section specifies the software architecture requirements and the software
architecture design at the second level of granularity - the benchmark service
of the benchmarking system. The benchmrk service will be responsible for monitor
program that will be run within a linux contatiner, which will monitor the users program.

\section{Overall Software Architecture}
\subsection{Architecture Requirements}
This section discusses the software architecture requirements around the
backend benchmark system infrastructure. The backend benchmark system will be
responsible for creating a linux container which will hold the monitor program
which will monitor the users program instance. In particular
the architecture requirements at the second level of granularity is specified.

The architecture requirements should specify:
\begin{itemize}
	\item the architectural responsibilities which need to be addressed
	\item the access and integration requirements for the system
	\item the quality requirements
\end{itemize}

Figure \ref{fig:benchmarkInfrastructure} show the a high-level infrastructure
view of the Benchmark service.

\begin{figure}[H]
  \begin{center}
  \includegraphics[scale=0.4]{../Diagrams and Charts/Overview/BenchmarkInfrastructure.jpg}
  \caption{A high-level overview of the software architecture for the Benchmark Service}
  \label{fig:benchmarkInfrastructure}
  \end{center}
\end{figure}


\subsubsection{Access and Integration Requirements}
\label{sec:accessIntegrationRequirementsManagementSystem}
\paragraph{Access Channels}
The access channel will only be through a messenging service to the managment service.
\begin{itemize}
	\item Messenging Service to Managment Service
\end{itemize}

\subparagraph{Messenging Service to Managment Service}
The system will expose a further access channel to be used by the so-called
"Management Client". This access channel will be
utilizing a message bus architecture to deliver messages between the benchmark
service and the management backend system.

This access channel will however not be accessable to end users.

\paragraph{Integration Channels}
The various integration channels of the benchmarking system
\begin{itemize}
	\item Integration with a persistence provider
	\item Integration with the human access channels, such as web and
		mobile interfaces
	\item Integration with the message bus architecture
	\item Integration with Linux Containers
\end{itemize}

The integration with the persistence provider is required as we need to persist
the measurements obtained with the benchmarking service.\\\\

The final intergration required by the benchmarking system is that of integration
with the message bus.

Once a program is uploaded via the a managment service, it will be deployed
to the benchmarking cluster in a Linux Container (LXC), upon which the
benchmarking will commence.  The reason for utilizing Linux containers is in
order to meet the security quality requirements as provided in
section \ref{sec:securityQualityRequirement}.

\subsubsection{Quality Requirements}
\label{sec:qualityRequirementManagementSystem}
The quality requirement are the requirements around the quality attributes of
the systems and the services it provides. This includes requirements like
maintainability, flexibility, extensibility, performance, scalability, security,
auditability, usability and testability requirements.
\paragraph{Security}
Security within the benchmarking service should concern the fact that user programs
are not allowed to manipulate the underlying system that it is running on.
Therefore we are utilizing Linux Containers.

\paragraph{Flexibility}
\paragraph{Maintainability}
Amongst the most important quality requirements for the system is
maintainability. It should be easy to maintain the system in the future. To this end

\begin{itemize}
\item future developers should be able to easily understand the system,
\item the technologies chosen for the system an be reasonably expected to be available for a long
time,
\item and developers should be able to easily and relatively quickly
	\begin{itemize}
		\item change aspects of the functionality the system provides, and
		\item add new functionality to the system.
	\end{itemize}
\end{itemize}

\paragraph{Scalability}
The system needs to be able to process multiple requests for benchmarking a program,
but not to hinder the results by running simultaneously. Therefore we need to
implement a scheduling algorithm which maximises throughput, without skewing results.

\paragraph{Testability}
All services offered by the system must be testable through
\begin{enumerate}
	\item automated unit tests testing components in isolation using mock objects, and
	\item automated integration tests where components are integrated within the actual environment.
\end{enumerate}

In either case, these functional tests should verify that
\begin{itemize}
	\item the service is provided if all pre-conditions are met (i.e. that no exception is raised)
	\item the correct execption is throw when the corresponding pre-condition
	is violated.
	\item that all post-conditions hold true once the service has been provided.
\end{itemize}

In addition to functional testing, the quality requirements should also be tested.

\paragraph{Integrability}
The system should be able to easily address future integration requirements
by providing access to its services using widely adopted public standards.

\paragraph{Deployability}
\label{sec:systemDeployability}

\section{Container}

\subsection{Architecture Requirements}
This section will discuss the architectural requirements of the containment
system in place.

\subsubsection{Quality Requirements}
The sections below will discuss the quality requirements the containment 
system will need to adhere to.
\paragraph{Flexibility}
The container needs to be able to handle multiple types of programs to benchmark
as well as being language agnostic. 

\paragraph{Maintainability}
The container environment should be self-maintaining after the initial setup of
the container, there should be no need for an administrator to maintain the container
environment.

\paragraph{Scalability}
Each container environment will only be running one program at a time.

\paragraph{Performance}
The use of a container in the system should impact performance negligibly, while
it will separate the program being tested from the system we hope that this will
not impact the performance of the benchmarking.  


\paragraph{Reliability}
The implementation of the container has the aim of increasing the reliability of
the system as a whole. This is because the container will separate the program
being benchmarked from the system as a whole and will nullify any malicious 
program from making changes to the system.

\subsubsection{Architectural Responsibilities}
The current responsibilities of the container environment are to separate the
testing environment from the system. 

\subsubsection{Architecture Constraints}
The chosen technology should be an accepted standard as well as being sufficiently
documented and supported

\subsection{Architecture Design}
\subsubsection{Tactics}

\subsubsection{Architectural Components}

\subsubsection{Frameworks and Technologies}
To implement the container environment we have decided to use \textbf{Docker} as
this containment environment is well tested, well documented, with wide support.
Docker is initialized from the start which is why it surpassed our other consideration
of using the native \textbf{Linux Containment} system, as this would require setup
needing in depth knowledge of a system we weren't comfortable with.

\paragraph{Concepts and Constraints for Application Components}


\section{Monitor}

\subsection{Architecture Requirements}


\subsubsection{Quality Requirements}
\paragraph{Flexibility}
The Monitor should designed in such a way that it can benchmark any program
written in any language. This includes programming and scripting languages.

\paragraph{Maintainability}


\paragraph{Scalability}


\paragraph{Performance}


\paragraph{Reliability}


\subsubsection{Architectural Responsibilities}


\subsubsection{Architecture Constraints}


\subsection{Architecture Design}
\subsubsection{Tactics}

\subsubsection{Architectural Components}

\subsubsection{Frameworks and Technologies}

\paragraph{Concepts and Constraints for Application Components}

