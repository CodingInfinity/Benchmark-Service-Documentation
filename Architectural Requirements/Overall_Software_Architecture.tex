This section specifies the software architecture requirements and the software architecture design
for the first level of granularity - the system as a whole. The output will be the high level software
architecture components, the infrastructure between them and the tactics used at the first level of
granularity to realize the quality requirements for the system.
Subsequent sections will focus on the software architecture requirements and design for these
high-level architectural components. Note that many of the architectural requirements (particularly
the quality requirements) will be propagated down to lower level architectural components.

\section{Architecture Requirements}
This section discusses the software architecture requirements around the 
required software infrastructure within which the application functionality is
to be developed. The purpose of this infrastructure is to address the
non-functional requirements. In particular the architecture requirements
specify.
\begin{itemize}
	\item the architectural responsibilities which need to be addressed
	\item the access and integration requirements for the system
	\item the quality requirements
	\item the architecture constraints specified by the client
\end{itemize}

\subsection{Architectural Scope}
In this section we discuss architectural responsibilities which need to be addressed by the software
architecture. These include:\\
\begin{itemize}
	\item Providing a persistance infrastructure to be used
	\item Providing a reporing infrastructure that will genretate reports from the persistance infrastructure
	\item Providing a secured enviroment in which to execute the programs that need to be benchmarked.
	In this case there are security risks since we are allowing users to upload code to the server which
	will be executed. Here Docker will be used. The executable code will packaged in a docker container
	and sent to the execution enviroment. This will ensure security since the code will only be able to
	"break" the docker container and will not be able to break out of the contianer and influence the external
	enviroment.
	\item A server which will accept the code to be benchmarked and do the above mentioned Dockerisation of the code
	before it is pushed to the execution enviroment.
	\item A web server which will serve the web interface.
	\item A server to expose REST resources which will be used by the web front end to access the backend services
\end{itemize}

\subsection{Access and Integration Requirements}
\subsubsection{Access Channels}
\subsubsection{Integration Channels}


\subsection{Quality Requirements}
This section will specify the quality requirements at the highest level of
granularity. These requirements will propagate throughout the entire system
into every other compopent. Quality requirements that are spesific to certain
parts of the system will be discussed under that system's section.

\subsection{Architectural Responsibilities}

\subsection{Architecture Constraints}

\label{sec:systemArchitecturalConstraints}
The chosen architecture was determined to best fulfill the non-functional
requirements for the system as well as the requirements set forth by the
client.
\begin{itemize}
	\item The architecture must be deployable on Linux servers
	\item All libraries, frameworks, programming languages and any other
	material of sort utlized within this project must be open-source.
	\item All libraries, frameworks, programming languages and any other
	material of sort used must be supported by open standards, have an 
	active and vibrant support community and should have an active 
	release cycle. 	
\end{itemize}

\subsection{Architecture Design}
\subsubsection{Tactics}
\subsubsection{Architectural Components}
\subsubsection{Frameworks and Technologies}
\subsubsection{Concrete Realization of Architectural Components}
\subsubsection{Concepts and Constraints for Application Components}
