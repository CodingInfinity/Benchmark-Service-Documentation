The system will be accessed via a web interface. This interface will access the
backend server via REST rescources that will be exposed using the Spring
Framework. The web front end will use AngularJS to access these RESTful services.\\ \\
Futher more there will be an integration channel between the watcher program that
does the actual benchmarking and the program that is being bechmarked.\\
In this regard two scenario's exist:
\begin{itemize}
	\item The fist option is that the "main" program that the user uploads is simply
	benchmarked from start to finish.
	\item The second option is that the user wants to bechmark only a part of the
	system program's execution. In this case the program being benchmarked and
	the watcher program will communicate through sockets to determine when the
	benchmarks should be started and stopped. (Here Sockets are used to communicate
	between operating system proccesses and not network nodes)
\end{itemize}
Furthermore once the program is uploaded via the web interface, it will be deployed
to the benchmarking server as a docker instance. This is to meet quality requirements
pertaining to security which is discussed under it's own section.
