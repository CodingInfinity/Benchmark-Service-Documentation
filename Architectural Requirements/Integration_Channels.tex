The various integration channels of the benchmarking system
\begin{itemize}
	\item Integration with a persistence provider
	\item Integration with the human access channels, such as web and
		mobile interfaces 
	\item Integration with the message bus architecture
	\item Integration with Linux Containers
\end{itemize}

The integration with the persistence provider is required as we need to persist
the measurements obtained with the benchmarking client. Further more as per the
client's request, test data must also be persisted as it is envision that a
repository of test data will be build.

In order to make the results from the benchmarking tests useful, users will
need some way to interact and manipulate the data in order to be able to
derive value from this data. To enable users to interact with the data,
two human access channels will be created, namely a web and mobile interface
to allow users to interact and derive value from the data.

The final intergration required by the management system is that of integration
with the message bus. In order to better decouple the benchmarking client and
management systems, a message bus architecture was introduced. The management
system will process the results from a queue structure managed by the messaging
system. 

Once a program is uploaded via the a human access channel, it will be deployed
to the benchmarking cluster in a Linux Container (LXC), upon which the
benchmarking will commence.  The reason for utilizing Linux containers is in
order to meet the security quality requirements as provided in
section \ref{sec:securityQualityRequirement}.

