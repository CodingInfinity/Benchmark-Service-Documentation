The application server provides the required execution environment for business
services to be deployed and executed within. The application server is responsible
for realizing the business layer.

\subsection{Architecture Requirements}
The architectural requirements include refined quality requirements and architecural
responsibilities. The applicatoin server however doesn't include additional
architecture constraints other than those specified in section \ref{sec:systemArchitecturalConstraints}

\subsubsection{Access and Integration Requirements}
The stated access and integration requirements in section \ref{sec:accessIntegrationRequirementsManagementSystem}
need to be propegated down to this lower level component. The application server
will be mainly responsible for exposing these channels realized by other lower
level components.

Further more stated quality requirements in section \ref{sec:qualityRequirementManagementSystem}
need to also be propegated down to this layer as much of the quality
requirements will realized by the application server. Further refinement of the
quality requirements is however provided.

\subsubsection{Quality Requirements}
\paragraph{Flexibility}
The application server should allow for the flexable deployment of any layer at
any-time with the minimal system downtime and no lose of information to occur
because the system is down. This is especially important as one doesn't want to
lose the results of users, should they be communicated back to the system
during a time which it is down.

\paragraph{Security}
The application should allow for both the authentication and autherization of
users. It is important to note the difference between these similar words.

Authentication is a systematic process of proof to determine whether a said
party is genuine.  Authorization is the verification of whether the said party
has the required privilidges to perform the requested action on or in a resource.

The application server should support authentication within some neutral framework
to allow for future expansion of authentication methods such as LDAP and in addition
should support authorization throught a role-based system. Users access to the
system should happen soley through the approved access channels as to prevent users
from directly accessing the datbase, message bus and other components directly.

\subsubsection{Architectural Responsibilities}
\subsubsection{Architecture Constraints}
\subsection{Architecture Design}
\subsubsection{Tactics}
\subsubsection{Architectural Components}
\subsubsection{Frameworks and Technologies}
\paragraph{Concrete Realization of Architectural Components}
\paragraph{Tactics}
\paragraph{Tools}
\paragraph{Concepts and Constraints for Application Components}