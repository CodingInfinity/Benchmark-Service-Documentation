\documentclass[11pt,a4paper]{article}

\usepackage{titling}
\usepackage[hidelinks]{hyperref}
\usepackage{graphicx}
\usepackage{grffile}
\usepackage{float}
\usepackage{geometry}
\usepackage{listings}

\newcommand{\subtitle}[1]{
  \posttitle{
    \par\end{center}
    \begin{center}\large#1\end{center}
    \vskip0.5em}
}

\begin{document}
\title{Benchmark Service Testing Documentation}
\subtitle{ Git: \url{https://github.com/CodingInfinity/Benchmark-Service-Documentation} \\ GitHub Organisation: \url{https://github.com/CodingInfinity}}
\begin{figure}
			\centering
			\includegraphics[height=230px]{../Images/CodingInfinity.png}
\end{figure}
\author{
	\textbf{The Client:} \\
	Ms Vreda Pieterse  \\
	Department of Computer Science \\
	University Of Pretoria
	\\
	\\
	\textbf{The Team:} \\
	Andrew Broekman		\emph{11089777}	\\
	Brenton Watt		\emph{14032644}	\\
	Fabio Loreggian		\emph{14040426}	\\
	Reinhardt Cromhout	\emph{14009936}	\\
}
\date{\textbf{May 2016}}

\maketitle
\thispagestyle{empty}
\pagebreak

\tableofcontents
\pagebreak

\section{Introduction}
This is the testing documentation for the Benchmark Service. It details
who each componenet of the the system is tested.

\section{Benchmark Management Server}
\subsection{Scope}
Every use case that is provided by the system will be tested.

\subsection{Assumptions / Risks}
\subsection{Test Approach}
Automated Unit tests will be used to tests will be used to test the use cases.
\subsection{Test Environment}
The JUnit and Mockito frameworks are used to create and execute the automated
unit tests.

\section{Benchmark Web Interface}
\subsection{Scope}
\subsection{Assumptions / Risks}
\subsection{Test Approach}
\subsection{Test Environment}

\section{Benchmark Instrumentation Application}
\subsection{Scope}
\subsection{Assumptions / Risks}
\subsection{Test Approach}
\subsection{Test Environment}

\end{document}
