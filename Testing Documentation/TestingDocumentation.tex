\documentclass[11pt,a4paper]{article}

\usepackage{titling}
\usepackage[hidelinks]{hyperref}
\usepackage{graphicx}
\usepackage{grffile}
\usepackage{float}
\usepackage{geometry}
\usepackage{listings}
\usepackage{xcolor}

\newcommand{\subtitle}[1]{
  \posttitle{
    \par\end{center}
    \begin{center}\large#1\end{center}
    \vskip0.5em}
}
\definecolor{mygreen}{rgb}{0,0.6,0}
\definecolor{mygray}{rgb}{0.5,0.5,0.5}
\definecolor{mymauve}{rgb}{0.58,0,0.82}
\definecolor{light-gray}{gray}{0.95}
\lstset{ %
  backgroundcolor=\color{light-gray},   % choose the background color; you must add \usepackage{color} or \usepackage{xcolor}
  basicstyle=\footnotesize,        % the size of the fonts that are used for the code
  breakatwhitespace=false,         % sets if automatic breaks should only happen at whitespace
  breaklines=true,                 % sets automatic line breaking
  captionpos=b,                    % sets the caption-position to bottom
  commentstyle=\color{mygreen},    % comment style
  deletekeywords={...},            % if you want to delete keywords from the given language
  escapeinside={\%*}{*)},          % if you want to add LaTeX within your code
  extendedchars=true,              % lets you use non-ASCII characters; for 8-bits encodings only, does not work with UTF-8
  frame=single,	                   % adds a frame around the code
  keepspaces=true,                 % keeps spaces in text, useful for keeping indentation of code (possibly needs columns=flexible)
  keywordstyle=\color{blue},       % keyword style
  language=Octave,                 % the language of the code
  otherkeywords={*,...},           % if you want to add more keywords to the set
  numbers=left,                    % where to put the line-numbers; possible values are (none, left, right)
  numbersep=5pt,                   % how far the line-numbers are from the code
  numberstyle=\tiny\color{mygray}, % the style that is used for the line-numbers
  rulecolor=\color{black},         % if not set, the frame-color may be changed on line-breaks within not-black text (e.g. comments (green here))
  showspaces=false,                % show spaces everywhere adding particular underscores; it overrides 'showstringspaces'
  showstringspaces=false,          % underline spaces within strings only
  showtabs=false,                  % show tabs within strings adding particular underscores
  stringstyle=\color{mymauve},     % string literal style
  tabsize=2,	                   % sets default tabsize to 2 spaces
  title=\lstname                   % show the filename of files included with \lstinputlisting; also try caption instead of title
}

\begin{document}
\title{Benchmark Service Testing Documentation}
\subtitle{ Git: \url{https://github.com/CodingInfinity/Benchmark-Service-Documentation} \\ GitHub Organisation: \url{https://github.com/CodingInfinity}}
\begin{figure}
			\centering
			\includegraphics[height=230px]{../Images/CodingInfinity.png}
\end{figure}
\author{
	\textbf{The Client:} \\
	Ms Vreda Pieterse  \\
	Department of Computer Science \\
	University Of Pretoria
	\\
	\\
	\textbf{The Team:} \\
	Andrew Broekman		\emph{11089777}	\\
	Brenton Watt		\emph{14032644}	\\
	Fabio Loreggian		\emph{14040426}	\\
	Reinhardt Cromhout	\emph{14009936}	\\
}
\date{\textbf{September 2016}}

\maketitle
\thispagestyle{empty}
\pagebreak

\tableofcontents
\pagebreak

\section{Introduction}
This is the testing documentation for the Benchmark Service. It details how each 
component of the system is tested.

\subsection{Purpose}
This document combines the unit test plan and report into a single coherent artifact.
The overall broad purpose of this system is to allow a user to upload
an algorithm and have it benchmarked with a dataset, the algorithm will be benchmarked according
to wall-clock time, CPU usage and memory usage. This document serves to illustrate the tests
we have implemented to ensure the system is working correctly. Test driven development is 
essential to our project as it allows us to know what areas of our project are working as they
should and which areas are either causing errors or not returning the expected results.

\subsection{Scope}
The scope of this document is structured as follows. The features that are considered for
testing are listed in section .... Tests that have been identified from the requirements are
discussed in detail in section .... Furthermore, this document outlines the test environment
and the risks involved in the testing approaches that will be followed. Assumptions and
dependencies of this test plan will also be mentioned. Section \ref{Results} and \ref{conclusion} outlines,
discusses and concludes on the results of the tests, respectively.

\subsection{Test Environment}
\label{sec:testEnvironment}
\begin{enumerate}
	\item Programming Languages:
		\begin{itemize}
			\item Java
		\end{itemize}
	\item Testing Frameworks:
		\begin{itemize}
			\item JUnit
			\item Mockito
			\item Powermock with Spring
		\end{itemize}
	\item Operating System
		\begin{itemize}
			\item Linux
		\end{itemize}
	\item Web Browsers
		\begin{itemize}
			\item Google Chrome
		\end{itemize}
\end{enumerate}

\subsection{Assumptions and Dependencies}
To run the tests, the following is assumed to be in place:
\begin{itemize}
  \item Cloned version of the Benchmark Management Repository
  \item Java 7+ JVM
  \item Maven 3.0.0+
\end{itemize}

There are no additional dependecies required, as Maven will download and setup
required dependencies to perform testing.

\section{Unit Test Plan}
\subsection{Benchmark Management Server}
\label{sec:benchmarkManagementServerUnitTestPlan}
The Benchmark Management Server is a Maven based project, and as such makes use
of the Maven framework to assist in executing tests. All tests for the system
can be located in the \texttt{com.codinginfinity.benchmark.management.test} package located
in the \texttt{/src/test/java/} folder.

As such, only some tests will be illustrated. The reader is however referred to the
source code for the remainder of the tests.

\subsubsection{Functional Feature to be Tested}
All functional use cases, as outlined in the document \texttt{Benchmark Service 
Functional Requirements Documentation}, have associated test implemented in the
system. The stated document also specifies required constraints, pre- and post-conditions
as well as any other specific detail to be tested within the relevant diagrams.

Unit tests for the use cases will be implemented as black and white box tests.

\section{Test Cases}
As all use cases in the system have implemented unit tests, the following
specifications will be propegated down to every unit test.

Tests cases for the following modules exists:
\begin{itemize}
  \item Experiment Management
  \item Notifications
  \item Reporting
  \item Repository Management
  \item User Management
\end{itemize}

\subsection{Objetive}
The purpose of the unit test is to ensure that:
\begin{itemize}
  \item Pre-conditions are checked and raise appropriate exceptions when not met
  \item Post conditions are fulfilled
  \item Stated functional requirements specific to the use is implemented
\end{itemize}

\subsection{Input}
For every use case, both valid and invalid entries will be construcuted to
ensure that exceptions are raised if pre-condiions are not met. Further all return
values are checked to ensure that the returned object is correct according the stated
functional requirements.

\subsection{Outcome}
Every use case should fulfill its stated service contract by returning the
appropriate respone object as outlined in the the functional requirements
document.

\section{Pass/Fail Criteria}
For every use case tested, the following criteria must be met for the test to be
considered a \textbf{Pass}.
\begin{itemize}
  \item All pre-conditions should be fulfilled. If not, an appropriate execption
    must be raised as to alert the client.
  \item All post-condtions must hold upon returning from the function call. If 
    a post condition doesn't hold, the test will fail.
  \item Any stated functionality specific to the use case as set out in the
    functional requirements document must be validated within the unit test to
    ensure stated functionality is implemented.
\end{itemize}

\section{Unit Test Report}
\subsection{Detailed Test Results}\label{Results}
\subsubsection{Overview of test results}
All 203 of the implemented unit tests in the Benchmark Management System passed
successfully as can be observed on \url{https://travis-ci.org/CodingInfinity/Benchmark-Management-Server}

Refer to section \ref{sec:testEnvironment} for a list of the frameworks used
to test the management system. These frameworks where selected as they are currently
the most widely used frameworks in Java, and because most open source projects utlize
these frameworks. It is important to select frameworks that are recoginzed by the
open source community, as the project is envisage to become an open source project.

\subsubsection{Functional requirements test results}
Refer to section \ref{sec:benchmarkManagementServerUnitTestPlan} on where the 
tests for the project in question can be located.

\section{Other}
\subsection{POM file for Benchmark Management Server}
\begin{lstlisting}[language=xml]
<?xml version="1.0" encoding="UTF-8"?>
<project xmlns="http://maven.apache.org/POM/4.0.0" xmlns:xsi="http://www.w3.org/2001/XMLSchema-instance" xsi:schemaLocation="http://maven.apache.org/POM/4.0.0 http://maven.apache.org/maven-v4_0_0.xsd">
  <modelVersion>4.0.0</modelVersion>

  <name>Benchmark Management Server</name>
  <description>Management backend and co-ordination platform for the benchmark service platform</description>
  <inceptionYear>2016</inceptionYear>

  <prerequisites>
    <maven>3.0.0</maven>
  </prerequisites>

  <parent>
    <artifactId>spring-boot-starter-parent</artifactId>
    <groupId>org.springframework.boot</groupId>
    <version>1.3.5.RELEASE</version>
    <relativePath />
  </parent>

  <groupId>com.codinginfinity.benchmark</groupId>
  <artifactId>management</artifactId>
  <version>0.1-SNAPSHOT</version>
  <packaging>war</packaging>

  <organization>
    <name>Department of Computer Science, University of Pretoria</name>
    <url>http://www.cs.up.ac.za</url>
  </organization>

  <licenses>
    <license>
      <name>GNU AFFERO GENERAL PUBLIC LICENSE Version 3, 19 November 2007</name>
      <url>https://www.gnu.org/licenses/agpl-3.0.txt</url>
    </license>
  </licenses>

  <scm>
    <url>scm:git:git@github.com:CodingInfinity/Benchmark-Management-Server.git</url>
    <connection>scm:git:git@github.com:CodingInfinity/Benchmark-Management-Server.git</connection>
    <developerConnection>scm:git:git@github.com:CodingInfinity/Benchmark-Management-Server.git</developerConnection>
  </scm>

  <developers>
    <developer>
      <id>AndrewBroekman@tuks.co.za</id>
      <name>Andrew Broekman</name>
      <organization>University of Pretoria</organization>
      <email>AndrewBroekman@tuks.co.za</email>
    </developer>

    <developer>
      <id>u14040426@tuks.co.za</id>
      <name>Fabio Loreggian</name>
      <organization>University of Pretoria</organization>
      <email>u14040426@tuks.co.za</email>
    </developer>

    <developer>
      <id>BrentonWatt@tuks.co.za</id>
      <name>Brenton Watt</name>
      <organization>University of Pretoria</organization>
      <email>BrentonWatt@tuks.co.za</email>
    </developer>

    <developer>
      <id>u14009936@tuks.co.za</id>
      <name>Reinhardt Cromhout</name>
      <organization>University of Pretoria</organization>
      <email>u14009936@tuks.co.za</email>
    </developer>
  </developers>

  <ciManagement>
    <system>Travis CI</system>
    <url>https://travis-ci.org/CodingInfinity/Benchmark-Management-Server</url>
  </ciManagement>

  <issueManagement>
    <system>Github Issues</system>
    <url>https://github.com/CodingInfinity/Benchmark-Management-Server/issues</url>
  </issueManagement>

  <properties>
    <activemq.version>5.13.3</activemq.version>
    <camel.version>2.17.1</camel.version>
    <commons-compress.version>1.11</commons-compress.version>
    <commons-lang.version>2.6</commons-lang.version>
    <commons-io.version>2.5</commons-io.version>
    <dbh2.version>1.4.192</dbh2.version>
    <hibernate.version>4.3.11.Final</hibernate.version>
    <jacoco-maven-plugin.version>0.7.6.201602180812</jacoco-maven-plugin.version>
    <java.version>1.8</java.version>
    <javax.inject.version>1</javax.inject.version>
    <javax.validation.version>1.1.0.Final</javax.validation.version>
    <junit.version>4.12</junit.version>
    <liquibase-hibernate4.version>3.5</liquibase-hibernate4.version>
    <liquibase-slf4j.version>1.2.1</liquibase-slf4j.version>
    <liquibase.version>3.4.2</liquibase.version>
    <lombok.version>1.16.8</lombok.version>
    <maven-enforcer-plugin.version>1.4.1</maven-enforcer-plugin.version>
    <maven.compiler.source>${java.version}</maven.compiler.source>
    <maven.compiler.target>${java.version}</maven.compiler.target>
    <mockito.version>2.0.2-beta</mockito.version>
    <powermock.version>1.6.5</powermock.version>
    <project.testresult.directory>${project.build.directory}/test-results</project.testresult.directory>
    <sortpom-maven-plugin.version>2.5.0</sortpom-maven-plugin.version>
    <spring-framework.version>4.2.0.RELEASE</spring-framework.version>
    <spring-security-oauth2.version>2.0.10.RELEASE</spring-security-oauth2.version>
    <springfox.version>2.5.0</springfox.version>
    <thrift.version>0.9.3</thrift.version>
  </properties>

  <dependencies>

    <!-- ActiveMQ Dependencies -->
    <dependency>
      <groupId>org.apache.activemq</groupId>
      <artifactId>activemq-broker</artifactId>
      <version>${activemq.version}</version>
    </dependency>
    <dependency>
      <groupId>org.apache.activemq</groupId>
      <artifactId>activemq-camel</artifactId>
      <version>${activemq.version}</version>
    </dependency>

    <!-- Apache Camel -->
    <dependency>
      <groupId>org.apache.camel</groupId>
      <artifactId>camel-amqp</artifactId>
      <version>${camel.version}</version>
    </dependency>
    <dependency>
      <groupId>org.apache.camel</groupId>
      <artifactId>camel-core</artifactId>
      <version>${camel.version}</version>
    </dependency>
    <dependency>
      <groupId>org.apache.camel</groupId>
      <artifactId>camel-spring-boot-starter</artifactId>
      <version>${camel.version}</version>
    </dependency>

    <!-- Apache Compress -->
    <dependency>
      <groupId>org.apache.commons</groupId>
      <artifactId>commons-compress</artifactId>
      <version>${commons-compress.version}</version>
    </dependency>

    <!-- Apache Thrift -->
    <dependency>
      <groupId>org.apache.thrift</groupId>
      <artifactId>libthrift</artifactId>
      <version>${thrift.version}</version>
    </dependency>

    <!-- Apache Commons Dependencies -->
    <dependency>
      <groupId>commons-lang</groupId>
      <artifactId>commons-lang</artifactId>
      <version>${commons-lang.version}</version>
    </dependency>
    <dependency>
      <groupId>commons-io</groupId>
      <artifactId>commons-io</artifactId>
      <version>${commons-io.version}</version>
    </dependency>

    <dependency>
      <groupId>io.springfox</groupId>
      <artifactId>springfox-swagger-ui</artifactId>
      <version>${springfox.version}</version>
    </dependency>

    <!-- H2 Database -->
    <dependency>
      <groupId>com.h2database</groupId>
      <artifactId>h2</artifactId>
      <version>${dbh2.version}</version>
    </dependency>

    <dependency>
      <groupId>com.fasterxml.jackson.dataformat</groupId>
      <artifactId>jackson-dataformat-csv</artifactId>
    </dependency>
    <dependency>
      <groupId>com.fasterxml.jackson.datatype</groupId>
      <artifactId>jackson-datatype-jsr310</artifactId>
    </dependency>

    <!-- Java EE Dependencies -->
    <dependency>
      <groupId>javax.inject</groupId>
      <artifactId>javax.inject</artifactId>
      <version>${javax.inject.version}</version>
    </dependency>
    <dependency>
      <groupId>javax.validation</groupId>
      <artifactId>validation-api</artifactId>
      <version>${javax.validation.version}</version>
    </dependency>

    <!-- Liquibase -->
    <dependency>
      <groupId>com.mattbertolini</groupId>
      <artifactId>liquibase-slf4j</artifactId>
      <version>${liquibase-slf4j.version}</version>
    </dependency>
    <dependency>
      <groupId>org.liquibase</groupId>
      <artifactId>liquibase-core</artifactId>
      <exclusions>
        <exclusion>
        <artifactId>jetty-servlet</artifactId>
          <groupId>org.eclipse.jetty</groupId>
        </exclusion>
      </exclusions>
    </dependency>

    <!-- Lombok -->
    <dependency>
      <groupId>org.projectlombok</groupId>
      <artifactId>lombok</artifactId>
      <version>${lombok.version}</version>
      <scope>provided</scope>
    </dependency>

    <!-- Springframework Dependencies -->
    <dependency>
      <groupId>org.springframework</groupId>
      <artifactId>spring-jms</artifactId>
    </dependency>
    <dependency>
      <groupId>org.springframework.security.oauth</groupId>
      <artifactId>spring-security-oauth2</artifactId>
      <version>${spring-security-oauth2.version}</version>
    </dependency>

    <!-- Spring Boot Dependencies -->
    <dependency>
      <groupId>org.springframework.boot</groupId>
      <artifactId>spring-boot-autoconfigure</artifactId>
    </dependency>
    <dependency>
      <groupId>org.springframework.boot</groupId>
      <artifactId>spring-boot-configuration-processor</artifactId>
      <optional>true</optional>
    </dependency>
    <dependency>
      <groupId>org.springframework.boot</groupId>
      <artifactId>spring-boot-starter-aop</artifactId>
    </dependency>
    <dependency>
      <groupId>org.springframework.boot</groupId>
      <artifactId>spring-boot-starter-data-jpa</artifactId>
    </dependency>
    <dependency>
      <groupId>org.springframework.boot</groupId>
      <artifactId>spring-boot-starter-data-elasticsearch</artifactId>
    </dependency>
    <dependency>
      <groupId>org.springframework.boot</groupId>
      <artifactId>spring-boot-starter-mail</artifactId>
    </dependency>
    <dependency>
      <groupId>org.springframework.boot</groupId>
      <artifactId>spring-boot-starter-thymeleaf</artifactId>
    </dependency>
    <dependency>
      <groupId>org.springframework.boot</groupId>
      <artifactId>spring-boot-starter-web</artifactId>
    </dependency>

    <!-- Swagger Dependencies -->
    <dependency>
      <groupId>io.springfox</groupId>
      <artifactId>springfox-swagger2</artifactId>
      <version>${springfox.version}</version>
      <exclusions>
        <exclusion>
        <artifactId>mapstruct</artifactId>
          <groupId>org.mapstruct</groupId>
        </exclusion>
      </exclusions>
    </dependency>


    <!-- Testing Framework Dependencies -->
    <dependency>
      <groupId>junit</groupId>
      <artifactId>junit</artifactId>
      <version>${junit.version}</version>
      <scope>test</scope>
    </dependency>
    <dependency>
      <groupId>net.trajano.commons</groupId>
      <artifactId>commons-testing</artifactId>
      <version>1.0.1</version>
      <scope>test</scope>
    </dependency>
    <dependency>
      <groupId>org.hsqldb</groupId>
      <artifactId>hsqldb</artifactId>
      <scope>test</scope>
    </dependency>
    <dependency>
      <groupId>org.mockito</groupId>
      <artifactId>mockito-all</artifactId>
      <version>${mockito.version}</version>
      <scope>test</scope>
    </dependency>
    <dependency>
      <groupId>org.powermock</groupId>
      <artifactId>powermock-api-mockito</artifactId>
      <version>${powermock.version}</version>
      <scope>test</scope>
    </dependency>
    <dependency>
      <groupId>org.powermock</groupId>
      <artifactId>powermock-core</artifactId>
      <version>${powermock.version}</version>
      <scope>test</scope>
    </dependency>
    <dependency>
      <groupId>org.powermock</groupId>
      <artifactId>powermock-module-junit4</artifactId>
      <version>${powermock.version}</version>
      <scope>test</scope>
    </dependency>
    <dependency>
      <groupId>org.springframework</groupId>
      <artifactId>spring-test</artifactId>
      <version>${spring-framework.version}</version>
      <scope>test</scope>
    </dependency>
    <dependency>
      <groupId>org.subethamail</groupId>
      <artifactId>subethasmtp</artifactId>
      <version>3.1.7</version>
      <scope>test</scope>
    </dependency>
  </dependencies>

  <build>
    <defaultGoal>spring-boot:run</defaultGoal>
    <plugins>
      <plugin>
        <groupId>com.github.ekryd.sortpom</groupId>
        <artifactId>sortpom-maven-plugin</artifactId>
        <version>${sortpom-maven-plugin.version}</version>
        <executions>
          <execution>
            <phase>verify</phase>
            <goals>
              <goal>sort</goal>
            </goals>
          </execution>
        </executions>
        <configuration>
          <sortProperties>true</sortProperties>
          <nrOfIndentSpace>4</nrOfIndentSpace>
          <sortDependencies>groupId,artifactId</sortDependencies>
          <sortPlugins>groupId,artifactId</sortPlugins>
          <keepBlankLines>true</keepBlankLines>
          <expandEmptyElements>false</expandEmptyElements>
        </configuration>
      </plugin>
      <plugin>
        <groupId>org.apache.maven.plugins</groupId>
        <artifactId>maven-eclipse-plugin</artifactId>
        <configuration>
          <downloadSources>true</downloadSources>
          <downloadJavadocs>true</downloadJavadocs>
        </configuration>
      </plugin>
      <plugin>
        <groupId>org.apache.maven.plugins</groupId>
        <artifactId>maven-enforcer-plugin</artifactId>
        <version>${maven-enforcer-plugin.version}</version>
        <executions>
          <execution>
            <id>enforce-versions</id>
            <goals>
              <goal>enforce</goal>
            </goals>
          </execution>
        </executions>
        <configuration>
          <rules>
            <requireMavenVersion>
              <message>You are running an older version of Maven. Benchmark Management Server requires at least Maven 3.0</message>
              <version>[3.0.0,)</version>
            </requireMavenVersion>
            <requireJavaVersion>
              <message>You are running an older version of Java.  Benchmark Management Server requires at least JDK ${java.version}</message>
              <version>[${java.version}.0,)</version>
            </requireJavaVersion>
          </rules>
        </configuration>
      </plugin>
      <plugin>
        <groupId>org.apache.maven.plugins</groupId>
        <artifactId>maven-surefire-plugin</artifactId>
        <configuration>
          <argLine>-Djava.security.egd=file:/dev/./urandom -Xmx256m ${surefireArgLine}</argLine>
          <runOrder>alphabetical</runOrder>
        </configuration>
      </plugin>
      <plugin>
        <groupId>org.eluder.coveralls</groupId>
        <artifactId>coveralls-maven-plugin</artifactId>
        <version>4.2.0</version>
      </plugin>
      <plugin>
        <groupId>org.jacoco</groupId>
        <artifactId>jacoco-maven-plugin</artifactId>
        <version>${jacoco-maven-plugin.version}</version>
        <executions>
          <execution>
            <id>pre-unit-test</id>
            <goals>
              <goal>prepare-agent</goal>
            </goals>
            <configuration>
              <propertyName>surefireArgLine</propertyName>
            </configuration>
          </execution>
          <execution>
            <id>post-unit-test</id>
            <phase>test</phase>
            <goals>
              <goal>report</goal>
            </goals>
          </execution>
          <execution>
            <id>default-report</id>
            <phase>prepare-package</phase>
            <goals>
              <goal>report</goal>
            </goals>
          </execution>
        </executions>
        <configuration>
          <excludes>
            <exclude>**/**Exception.class</exclude>
            <exclude>**/aop/**/*.class</exclude>
            <exclude>**/config/**/*.class</exclude>
            <exclude>**/domain/**/*.class</exclude>
            <exclude>**/dto/**/*.class</exclude>
            <exclude>**/http/**/*.class</exclude>
            <exclude>**/jackson/mixin/**/*.class</exclude>
            <exclude>**/service/**/exception/*.class</exclude>
            <exclude>**/service/**/response/*.class</exclude>
            <exclude>**/service/**/request/*.class</exclude>
            <exclude>**/thrift/**/*.class</exclude>
            <exclude>**/web/rest/**/*.class</exclude>
          </excludes>
        </configuration>
      </plugin>
      <plugin>
        <groupId>org.springframework.boot</groupId>
        <artifactId>spring-boot-maven-plugin</artifactId>
        <configuration>
          <executable>true</executable>
          <arguments>
            <argument>--spring.profiles.active=dev</argument>
          </arguments>
        </configuration>
      </plugin>
      <plugin>
        <groupId>org.liquibase</groupId>
        <artifactId>liquibase-maven-plugin</artifactId>
        <version>${liquibase.version}</version>
        <configuration>
          <changeLogFile>src/main/resources/config/liquibase/master.xml</changeLogFile>
          <diffChangeLogFile>src/main/resources/config/liquibase/changelog/${maven.build.timestamp}_changelog.xml</diffChangeLogFile>
          <driver>org.postgresql.Driver</driver>
          <url>jdbc:postgresql://localhost:5432/postgresql</url>
          <defaultSchemaName></defaultSchemaName>
          <username>postgresql</username>
          <password></password>
          <referenceUrl>hibernate:spring:com.mycompany.myapp.domain?dialect=org.hibernate.dialect.PostgreSQL82Dialect&amp;hibernate.ejb.naming_strategy=org.springframework.boot.orm.jpa.hibernate.SpringNamingStrategy</referenceUrl>
          <verbose>true</verbose>
          <logging>debug</logging>
          <diffExcludeObjects>oauth_access_token, oauth_approvals, oauth_client_details, oauth_client_token, oauth_code, oauth_refresh_token</diffExcludeObjects>
        </configuration>
        <dependencies>
          <dependency>
            <groupId>org.javassist</groupId>
            <artifactId>javassist</artifactId>
            <version>3.18.2-GA</version>
          </dependency>
          <dependency>
            <groupId>org.liquibase.ext</groupId>
            <artifactId>liquibase-hibernate4</artifactId>
            <version>${liquibase-hibernate4.version}</version>
          </dependency>
          <dependency>
            <groupId>org.springframework.boot</groupId>
            <artifactId>spring-boot-starter-data-jpa</artifactId>
            <version>${project.parent.version}</version>
          </dependency>
        </dependencies>
      </plugin>
      <plugin>
        <groupId>com.spotify</groupId>
        <artifactId>docker-maven-plugin</artifactId>
        <version>0.4.5</version>
        <configuration>
          <imageName>benchmark</imageName>
          <dockerDirectory>src/main/docker</dockerDirectory>
          <resources>
            <resource>
              <targetPath>/</targetPath>
              <directory>${project.build.directory}</directory>
              <include>${project.build.finalName}.war</include>
            </resource>
          </resources>
        </configuration>
      </plugin>
    </plugins>
  </build>

  <reporting>
    <plugins>
      <plugin>
        <groupId>org.apache.maven.plugins</groupId>
        <artifactId>maven-surefire-report-plugin</artifactId>
        <version>2.19</version>
      </plugin>
    </plugins>
  </reporting>

  <profiles>
    <profile>
      <id>dev</id>
      <activation>
        <activeByDefault>true</activeByDefault>
      </activation>
      <properties>
        <logback.loglevel>DEBUG</logback.loglevel>
      </properties>
    </profile>

    <profile>
      <id>prod</id>
      <build>
        <plugins>
          <plugin>
            <groupId>org.springframework.boot</groupId>
            <artifactId>spring-boot-maven-plugin</artifactId>
            <configuration>
              <executable>true</executable>
              <arguments>
                <argument>--spring.profiles.active=prod</argument>
              </arguments>
            </configuration>
          </plugin>
        </plugins>
      </build>
      <properties>
        <logback.loglevel>INFO</logback.loglevel>
      </properties>
    </profile>
  </profiles>
</project>
\end{lstlisting}

\subsection{Conclusion}\label{conclusion}
The Benchmark Instrumentation Application and Benchmark Web Application was not
tested as no mature and standard way exsists to implement unit testing within the
said frameworks or languages. The difficulty encountered in the testing a Web Application
is that the application only issues RESTful calls to the backend. As such no logic
remain on the client side to be tested, which improves software reliability and security.

Further the Instrumentation software is built as a monolithic application, with
developer optimized code and routines as to elimate any possible side effects
which application may introduce. This optimization by the developers make software
very hard to test, as no component can really be tested in isolation.

\end{document}
