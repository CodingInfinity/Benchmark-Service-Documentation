\documentclass[11pt,a4paper]{article}

\usepackage{titling}
\usepackage[hidelinks]{hyperref}
\usepackage{graphicx}
\usepackage{grffile}
\usepackage{float}
\usepackage{geometry}
\usepackage{listings}

\newcommand{\subtitle}[1]{
  \posttitle{
    \par\end{center}
    \begin{center}\large#1\end{center}
    \vskip0.5em}
}

\begin{document}
\title{Benchmark Service Testing Documentation}
\subtitle{ Git: \url{https://github.com/CodingInfinity/Benchmark-Service-Documentation} \\ GitHub Organisation: \url{https://github.com/CodingInfinity}}
\begin{figure}
			\centering
			\includegraphics[height=230px]{../Images/CodingInfinity.png}
\end{figure}
\author{
	\textbf{The Client:} \\
	Ms Vreda Pieterse  \\
	Department of Computer Science \\
	University Of Pretoria
	\\
	\\
	\textbf{The Team:} \\
	Andrew Broekman		\emph{11089777}	\\
	Brenton Watt		\emph{14032644}	\\
	Fabio Loreggian		\emph{14040426}	\\
	Reinhardt Cromhout	\emph{14009936}	\\
}
\date{\textbf{May 2016}}

\maketitle
\thispagestyle{empty}
\pagebreak

\tableofcontents
\pagebreak

\section{Introduction}
This is the testing documentation for the Benchmark Service. It details how each 
component of the system is tested.

\subsection{Purpose}
This document combines the unit test plan and report into a single coherent artifact.
The overall broad purpose of this system is to allow a user to upload
an algorithm and have it benchmarked with a dataset, the algorithm will be benchmarked according
to wall-clock time, CPU usage and memory usage. This document serves to illustrate the tests
we have implemented to ensure the system is working correctly. Test driven development is 
essential to our project as it allows us to know what areas of our project are working as they
should and which areas are either causing errors or not returning the expected results.

\subsection{Scope}
The scope of this document is structured as follows. The features that are considered for
testing are listed in section .... Tests that have been identified from the requirements are
discussed in detail in section .... Furthermore, this document outlines the test environment
and the risks involved in the testing approaches that will be followed. Assumptions and
dependencies of this test plan will also be mentioned. Section \ref{Results} and \ref{conclusion} outlines,
discusses and concludes on the results of the tests, respectively.

\subsection{Test Environment}
\begin{enumerate}
	\item Programming Languages:
		\begin{itemize}
			\item Java
			\item C++
			\item HTML
			\item TypeScript
		\end{itemize}
	\item Testing Frameworks:
		\begin{itemize}
			\item JUnit
			\item Mockito
			\item Powermock with Spring
		\end{itemize}
	\item Coding Environment
		\begin{itemize}
			\item IntelliJ
			\item WebStorm
			\item CLion
		\end{itemize}
	\item Operating System
		\begin{itemize}
			\item Linux
		\end{itemize}
	\item Web Browsers
		\begin{itemize}
			\item Google Chrome
		\end{itemize}
\end{enumerate}

\subsection{Assumptions and Dependencies}
For a user to run the tests described below we assume that they have checked out our
repository and have a working maven setup with working JVM. Dependencies will be 
pulled with Maven, so there is no need for manual setup of dependencies.

\section{Unit Test Plan}
\subsection{Benchmark Management Server}
\subsubsection{Functional Feature to be Tested}


\subsection{Benchmark Web Interface}


\subsection{Benchmark Instrumentation Application}


\section{Unit Test Report}
\subsection{Detailed Test Results}\label{Results}
\subsubsection{Overview of test results}
\subsubsection{Functional requirements test results}
\subsection{Other}\label{other}
\subsection{Conclusion}\label{conclusion}

\end{document}
